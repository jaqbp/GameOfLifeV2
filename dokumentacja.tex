\documentclass[12pt]{article}
\usepackage[utf8]{inputenc}
\usepackage{polski}
\usepackage{setspace}
\usepackage{pdfpages}
\usepackage[T1]{fontenc}
\onehalfspacing

\title{Projekt: Gra w życie (Game of Life) w Pythonie}
\author{Piotr Skowroński, Jakub Pietrasik, Krzysztof Czuba}
\date{\today}

\begin{document}

\maketitle

\section{Skład zespołu oraz role w projekcie}
   - Projektant i programista: Piotr Skowroński\\
   - Programista i optymalizator: Krzysztof Czuba\\
   - Programista i tester: Jakub Pietrasik

\section{Opis tematyki projektu}
   Projekt dotyczy implementacji znanej gry w życie, znaną jako "Game of Life". Gra ta jest automatem komórkowym stworzonym przez Johna Conwaya, w którym komórki na planszy ewoluują na podstawie pewnych reguł. Celem gry jest badanie zachowań i ewolucji populacji komórek na dwuwymiarowej planszy.

\section{Opis realizacji i sposobu osiągnięcia celu projektowego}
   - Inicjalizacja planszy i parametrów gry, takich jak rozmiar planszy, kolory, prędkość itp.\\
   - Implementacja metody \texttt{draw\_grid}, która rysuje siatkę planszy.\\
   - Implementacja metody \texttt{draw\_cells}, która rysuje komórki na planszy.\\
   - Implementacja metody \texttt{get\_neighbour\_count}, która zlicza sąsiadujące komórki dla danej komórki.\\
   - Wyświetlanie ekranu z instrukcjami dla użytkownika.
   - Główna pętla gry z obsługą zdarzeń takich jak ruch myszy, pauza, zmiana prędkości itp.\\
   - Logika gry, w której komórki ewoluują na podstawie reguł.

\section{Bibliografia wraz z cytatowaniami w treści projektu}
   - John H. Conway, "The Game of Life," Scientific American, 1970.\\
   - Dokumentacja biblioteki Pygame: https://www.pygame.org/docs/

\section{Temat projektu wybrany z listy tematów}
   Projekt opiera się na wybranym temacie z listy związanym z implementacją gry w życie (temat 6) w języku Python przy użyciu biblioteki Pygame.

\end{document}

